\documentclass[12pt,a4j]{jreport}
\setcounter{secnumdepth}{5}
\usepackage[dvipdfmx]{graphicx}
\usepackage{amsmath,amssymb}
%\usepackage{comment}
\usepackage{graphicx}
%\usepackage{here}
\usepackage{bm}
\usepackage{url}
\usepackage{color}

\renewcommand{\baselinestretch}{1.5}

\renewcommand{\bibname}{参考文献}
\newcommand{\todo}[1]{{\bf\color{red}{\{Todo: #1\}}}}




\begin{document}


%%%%%%%%%%%%%%%%%%%%%%%%%%%%%%%%%%%%%
% 表紙
%%%%%%%%%%%%%%%%%%%%%%%%%%%%%%%%%%%%%
\begin{titlepage}

\begin{center}

    \vspace*{2cm}
    \Large 2025 年度 芝浦工業大学 工学部 情報工学科\\

    \vspace*{1.0cm}
    \Huge 卒 \qquad 業 \qquad 論 \qquad 文\\
    \vspace*{2.5cm}

    %TODO 編集 : 題目
    \Large CUIベーステキストエディタ配色生成システムへの\\GUI導入とユーザビリティ評価\\
    
    \vspace{4cm}
    \begin{tabular}{ll}
        %TODO 編集 : 題目
        \vspace*{2mm}
        学籍番号 & \qquad $\mathbf{AL22020}$ \\
        \vspace*{2mm}
        氏\phantom{  }名 & \qquad 伊藤 \quad 柊生   \\
        \vspace*{2mm}
        指導教員           & \qquad 篠埜 \quad 功     \\
        \vspace*{2mm}
        提出日             & \qquad 2026年2月...日
    \end{tabular}
\end{center}
\end{titlepage}



\begin{abstract}
現在、様々な項目についてカスタマイズ可能なテキストエディタが人々に広く使われており,その中でもテキストエディタの配色を自分の好みに変更するユーザーも多く存在する。
配色の変更には、テキストエディタの拡張機能やインターネット上で公開されているものを利用することができるが,好みに合った配色を探すのには時間がかかる.
また,自分で配色を独自に作成することも可能であるが,設定項目が非常に多い場合,自分の好みに合った配色を作成するのは時間がかかる.
本研究では,ユーザーの好みを反映しつつ,視認性の高いテキストエディタの配色を簡易的に生成する手法を提案する.本手法は
\end{abstract}


{\makeatletter
\let\ps@jpl@in\ps@empty
\makeatother
\pagestyle{empty}
\tableofcontents
\clearpage}

\setcounter{page}{1} 
\pagestyle{plain}

%%%%%%%%%%%%%%%%%%%%%%%%%%%%%%%%%%%%%%%%%%%%%%%%%%%%%%%%%%%%%
%%%%%%%%%%%%%%%%%%%%%%%%%%%%%%%%%%%%%%%%%%%%%%%%%%%%%%%%%%%%%
\chapter{はじめに}

%\section{背景}
%背景で去年の卒論を引用して触れる。
Visual Studio Code や Vim など様々な項目を設定可能な高機能テキストエディタがいたるところで多く使われている。
Stack overflow annual developer survey\cite{rateenv}の開発環境調査によると、2024年5月の調査では、図\ref{figure:rate_developenv}のように、質問に回答した開発者54,175人のうち73.6\%がVisual Studio Codeを使用している。

\begin{figure}[tb]
    \includegraphics[width=1\linewidth]{rate_developenv.png}
    \caption{\label{figure:rate_developenv}各テキストエディタの使用率}
\end{figure}

これらのテキストエディタの多くは、プログラムの視認性を高めるために、字句の種類ごとに色分けを行う機能を備えており、ユーザーはこれらの色設定を自由にカスタマイズすることが可能である。
テキストエディタにおける色設定の手段は大きく二つに分類できる。
第一の手段は、既に作成された配色データをダウンロードして利用する方法である。
Visual Studio Codeにおいては、テーマを拡張機能としてインストールすることで導入が可能であり、Visual Studio Marketplaceには、約37,000件に及ぶ拡張機能が登録されている。
各拡張機能は一つ以上のテーマを提供しており、ユーザーはその中から好みの配色を選択する。
しかしながら、これらの提供されているテーマの数が膨大であるため、希望に合致する配色を見つけ出す作業は容易ではない。
第二の手段は、ユーザー自身が個別に色設定を行う方法である。
Visual Studio Codeにおいては、ワークベンチ関連の色設定項目が約780項目ある\cite{colorbench}。
なお、ワークベンチの色設定においては、項目間に多くの依存関係があり、他の設定項目の色をもとに自動計算される項目も多くある。

このように、全ての設定項目を網羅的に記述することにより、自分好みの配色テーマを作成することはできるが、設定すべき項目数が多く、さらにそれぞれに対して適切な色を選定する必要があるため、作業には多大な労力を要する。
一方でテキストエディタの配色に関する研究は吉越ら\cite{sotu24}の研究を除いて筆者らが調べた限り見当たらない。

%\section{目的}
吉越らの研究で推薦された手法では、ユーザがRGB形式の入力によって指定した色に基づいて、テキストエディタの配色ファイルを生成し、その後、ユーザが納得のいくまで修正を繰り返し、好みの配色を生成することができる。
しかし、CUIのみを用いたツールであるため、ユーザーにとって不便であるという課題がある\cite{sotu24}。
そこで本研究ではユーザーがより容易に好みのテキストエディタの配色の設定行えるようにするため、GUIを導入することでツールの改善を行う。

%\section{内容}
%卒論で何をするかも書く。
GUIの導入は、まず、吉越らのツール\cite{sotu24}ではCLIで行われていて不便
であった文字色の選択の部分に対し、カラーピックを用いることにより行う。
また、変更後の配色は吉越らのツール\cite{sotu24}では生成した配色をVSCodeに反映させない限りツールの使用者は見ることができなかったが、
変更後の配色を適用前にサンプルの文字に色をつけることで、色選択と同じ画面で
確認できるようにツールの改善を行う。

改善を加えたツールの有効性を示すため、既存のツールと改善を加えたツールを
被験者に使用してもらう実験を行う。ツール使用後にsystem usability scale\cite{sus1996}でアンケート評価を行った結果、....\todo{実験後にうめる}。

%\section{本論文の構成}
本論文の構成は次のようになっている.
まず、\ref{chap:related}章で本研究の関連研究について述べる.
次に\ref{chap:method}章では提案手法について述べる。
さらに\ref{chap:experiment}章ではツールの比較実験について述べ、
最後に\ref{chap:conclusion}章で結論と今後の課題について述べる.

%%%%%%%%%%%%%%%%%%%%%%%%%%%%%%%%%%%%%%%%%%%%%%%%%%%%%%%%%%%%%
%%%%%%%%%%%%%%%%%%%%%%%%%%%%%%%%%%%%%%%%%%%%%%%%%%%%%%%%%%%%%
\chapter{関連研究}\label{chap:related}

\todo{次回2章以降も見る。}
これまでテキストエディタにおける配色生成に関する研究は、筆者の知る限り、吉越\cite{sotu24}らの研究を除き、行われていない。
一方で、テキストエディタの配色に近い研究として、Webページの配色生成システムに関する研究\cite{webcolor}などがあり、視認性を高くすることを目的に配色を生成するといった手法を論じている。
さらに、インターフェースの改良には、あらゆる評価手法を用い、それらを比較、評価することが重要であり、これについて様々な研究\cite{guivscui, guivscli, nielsen1994usability, sus1996}が行われてきた。
以下では、上記の内容について論じる。  

\section{テキストエディタにおける配色の自動生成}
吉越\cite{sotu24}\todo{情報処理学会論文誌にそのうち掲載されるので、掲載後はそれで置き換える}は、テキストエディタにおけるユーザの好みを反映した視認性の高い配色を推薦する手法を提案した。
この研究では、ユーザが好みの一色を一つの文字色として指定し、その色に対し焼きなまし法を用いることで、視認性が高くなるような背景色やほかの文字色を自動生成し、visual studio codeの配色設定ファイルとして出力する手法を提案した。
また、ユーザが望む場合、変更したい色を指定することで、生成した配色の微調整するができる。
さらに、手動で配色を作成した場合と提案した手法を用いたツールで配色を生成した場合における作成時間、配色の好み、配色の見やすさを比較する実験を行った結果、この手法の有用性が確認できた。
しかしながら、視認性が非常に悪い配色が生成される場合があることと、CUIのみを用いたツールのため不便であることが、課題であると述べている。

\section{webページにおける配色の自動生成}
藤本ら\cite{webcolor}は、マルチエージェント強化学習が用い、webページの配色生成の自動化のためのシステムを作成した。
配色の視認性を高めるために、Web content accessibility guidelines\cite{colorguide} において推奨される、webページの背景色と前景色とのコントラスト比の制約を利用した。
また、さらなる配色の最適化のために、生成された配色に対し、ユーザが評価を行い、それらを新たにシステムにフィードバックとして与える方法を用いている。


\section{インターフェースの優劣}
Gaff\cite{guivscui}は様々な性格タイプの被験者を集め、著者名やタイトルの検索課題をCUIとGUIでそれぞれ行ってもらい、作業におけるやりやすさやストレスの程度を評価する実験を行った。
実験結果はすべての被験者がGUIを使いやすかったと回答した一方で、6割の被験者がCUIの利用にストレスを感じたと回答したことでGUIの有意性を示した。

また、Feiziら\cite{guivscli}は、GUIとCLIどちらかのみを用いたタスクや二つを併用したタスクを用意し、17人のインターフェースデザイナーと15人のソフト開発者の被験者にそれらを実行させる実験を行った。
実験では複数の項目からなるユーザアンケートによる評価とタスクの成功率や平均のタスク時間を計測し、比較を行った。
実験結果はGUIを用いたタスクの成功率は100\%だったのに対し、CLIを用いたタスクの成功率は91.93\%であった。
また,タスクの平均時間もGUIを用いたタスクの方が短く、タスク終了後のユーザアンケートでは、画面デザインやソフトの機能性といった項目の評価平均点が高かったのに対し、「スクリプト入力のわかりやすさ」といったCLIに関する項目の評価平均点が低かった。
このことから実験で用いたソフト、Adobe Flash CL4においては、CLIよりもGUIのほうが作業を行うことに優れているインターフェースであるということが示された。

\section{ユーザビリティの評価方法}
ユーザビリティとはJIS規格において「特定のユーザが特定の利用状況において,システム,製品又はサービスを利用する際に,効果,効率及び満足を伴って特定の目標を達成する度合い」と定義されており\cite{ISO.9241-11:2018}、これを評価することは、インターフェースの改善において重要である。

Nilesen\cite{nielsen1994usability}は、ユーザビリティを評価するための方法として、ユーザビリティの専門家が評価を行う、ユーザビリティインスペクションを提言した。
その主な手法には、専門家が10のユーザビリティ原則に基づいて判断するヒューリスティック評価や、ユーザーの問題解決プロセスをシミュレートし、各ステップでのユーザーの目標と記憶内容が次の正しいアクションにつながるかどうかを確認するコグニティブ・ウォークスルーなどを含む、6つの手法がある。

さらにほかのユーザビリティ評価方法として、Brooke\cite{sus1996}が提言した、system usability scaleがある。system usability scaleはユーザビリティの専門家ではなく、システムのユーザーが図\ref{figure:sus}のシステムの仕様に関する10項目の質問に対して、5段階のリッカート尺度で回答し、それらを決まったルールに従って計算を行い、算出されたスコアを評価とする。

\begin{figure}[tb]
    \includegraphics[width=1\linewidth]{sus.png}
    \caption{\label{figure:sus}system usability scale における10項目の質問}
\end{figure}

\section{UI設計}

%%%%%%%%%%%%%%%%%%%%%%%%%%%%%%%%%%%%%%%%%%%%%%%%%%%%%%%%%%%%%
%%%%%%%%%%%%%%%%%%%%%%%%%%%%%%%%%%%%%%%%%%%%%%%%%%%%%%%%%%%%%
\chapter{提案手法}\label{chap:method}
本研究ではJIS規格に基づいた、CIE 1976 L*a*b*色空間\cite{ISO.11664-4:2019}およびRGB色空間を用いる。
またこのRGB色空間というのはsRGB空間\cite{ISO.11664-6:2022}を示すものとする。

%%%%%%%%%%%%%%%%%%%%%%%%%%%%%%%%%%%%%%%%%%%%%%%%%%%%%%%%%%%%%
%%%%%%%%%%%%%%%%%%%%%%%%%%%%%%%%%%%%%%%%%%%%%%%%%%%%%%%%%%%%%
\chapter{実験}\label{chap:experiment}
本章では、\ref{chap:method}章で提案したGUI導入後ツールと、既存のCUIベースツールを用いて、Visual Studio Codeを使用したことがある人々を対象に、それぞれのツールで場合にユーザビリティにどのような違いがでるのか検証する実験について記述する。

\section{実験手順}
はじめに、被験者は既存のCUIベースツールを用いて、テキストエディタの配色作成を行う。
このツールはユーザがコマンドライン引数を用いて、カラーコードで好みの色を入力することによって、自動で色に基づいて配色を生成し、設定ファイルを作成する。
その後、ユーザは設定ファイルをVisual Studio Codeで反映させ、さらに調整したい色をコマンドラインでコマンドを用いて選択し、満足のいく配色を作成する。
これを制限時間60分で行い、時間内に満足のいく配色を生成できなかった場合でも終了とする。
時間計測は、最初の色を指定するためにcssファイルを用いて、カラーピッカーを開いた時点から最終的な配色がVisual Studio Codeに反映されるまでの時間を計測する。
その後被験者はsystem usability scale の10項目の質問に回答する。

次に、改良後のGUI導入ツールを用いて、テキストエディタの配色作成を行う。
このツールは\todo{具体的な実装が決まってから埋める}
これを制限時間60分で行い、時間内に満足のいく配色を生成できなかった場合でも終了とする。
時間計測は、最初の色を指定するためにカラーピッカーを開いた時点から最終的な配色がVisual Studio Codeに反映されるまでの時間を計測する。
その後被験者は system usability scale の10項目の質問に回答する。

\section{実験結果}
それぞれのツールにおいて、配色生成にかかった平均時間とsystem usability scaleのスコアは以下のようになった。

%%%%%%%%%%%%%%%%%%%%%%%%%%%%%%%%%%%%%%%%%%%%%%%%%%%%%%%%%%%%%
%%%%%%%%%%%%%%%%%%%%%%%%%%%%%%%%%%%%%%%%%%%%%%%%%%%%%%%%%%%%%
\chapter{結論と今後の課題}\label{chap:conclusion}

\chapter*{謝辞}
\addcontentsline{toc}{chapter}{謝辞}

\pagebreak
\addcontentsline{toc}{chapter}{参考文献}
\bibliographystyle{junsrt}
\bibliography{ref.bib}


\end{document}
