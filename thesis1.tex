\documentclass[12pt,a4j]{jreport}
\setcounter{secnumdepth}{5}
\usepackage[dvipdfmx]{graphicx}
\usepackage{amsmath,amssymb}
\usepackage{comment}
\usepackage{graphicx}
\usepackage{here}
\usepackage{bm}
\usepackage{url}

\renewcommand{\baselinestretch}{1.5}

\renewcommand{\bibname}{参考文献}





\begin{document}


%%%%%%%%%%%%%%%%%%%%%%%%%%%%%%%%%%%%%
% 表紙
%%%%%%%%%%%%%%%%%%%%%%%%%%%%%%%%%%%%%
\begin{titlepage}

\begin{center}

    \vspace*{2cm}
    \Large 2025 年度 芝浦工業大学 工学部 情報工学科\\

    \vspace*{1.0cm}
    \Huge 卒 \qquad 業 \qquad 論 \qquad 文\\
    \vspace*{2.5cm}

    %TODO 編集 : 題目
    \Large 好みを反映した視認性の高いテキストエディタの配色を\\推薦する手法の提案および実装の\\引継ぎ研究
    
    \vspace{4cm}
    \begin{tabular}{ll}
        %TODO 編集 : 題目
        \vspace*{2mm}
        学籍番号 & \qquad $\mathbf{AL22020}$ \\
        \vspace*{2mm}
        氏\phantom{  }名 & \qquad 伊藤 \quad 柊生   \\
        \vspace*{2mm}
        指導教員           & \qquad 笹埜 \quad 功
    \end{tabular}
\end{center}
\end{titlepage}



\begin{abstract}
概要
\end{abstract}


{\makeatletter
\let\ps@jpl@in\ps@empty
\makeatother
\pagestyle{empty}
\tableofcontents
\clearpage}

\setcounter{page}{1} 
\pagestyle{plain}

%%%%%%%%%%%%%%%%%%%%%%%%%%%%%%%%%%%%%%%%%%%%%%%%%%%%%%%%%%%%%
% 序論 
%%%%%%%%%%%%%%%%%%%%%%%%%%%%%%%%%%%%%%%%%%%%%%%%%%%%%%%%%%%%%
\chapter{序論(この章タイトルは一例.各自内容に合わせてつけること)}

\section{背景}
はいけい

\subsection{図表参照の例}
1-1-1

\begin{figure}[ht]
	\centering
	\includegraphics[keepaspectratio, width=120mm]{img/sample.png}
	\caption{提案法に用いた3層のニューラルネットワーク.キャプションにはこの図の説明を書く.}
	\label{fig_nn}
\end{figure}

\begin{table}[ht]
  \caption{手法Aおよび手法Bの正解率と平均計算時間.}
  \label{table_a}
  \centering
  \begin{tabular}{lcr}
    \hline
    手法   & 正解率[\%]  &  計算時間[ms]  \\
    \hline \hline
    手法A  & 92.3  & 512 \\
    手法B  & 87.4  & 32  \\
    \hline
  \end{tabular}
\end{table}



\subsection{関連研究参照の例}
1-1-2

\section{Latexファイルのコンパイル方法}

\subsection{ローカルにLatex環境を構築する場合}
各自好みの環境を使ってコンパイルするとよい.例として,TeXLiveを利用す場合は以下の手順でpdfを構築できる.
(1) TexLive\cite{TexLive}をインストール.
(2) フォルダ内のresume.texをTexLiveと同梱されているTexWorksで開く.
(3) 左上で『pBibTex』を指定しコンパイルを実行.
(4) 左上で『pLatex』を指定しコンパイルを実行(参考文献が?となる場合はこれを複数回実行).


\subsection{Overleafを利用する場合}
Overleafを利用する場合,以下の手順でpdfを作成できる.
(1)フォルダ内のresume.tex, ref.bib, latexmkrc, img/sample.pngをoverleafのプロジェクトにコピー.
(2)OVerleafのメニューより,コンパイラを『LaTeX』に,TexLive versionを2020に変更.
(3)リコンパイルボタンを押す.
もしエラーが起こる場合,(*)右側の画面からキャッシュファイルを削除する,(*)コンパイラを違うものに変更してコンパイルしてから再度LaTeXでコンパイル,(*)フォルダをつくりその中でresume.texをコンパイル,などを試すとうまくいく場合がある.



\section{目的}



\section{本論文の構成}



%%%%%%%%%%%%%%%%%%%%%%%%%%%%%%%%%%%%%%%%%%%%%%%%%%%%%%%%%%%%%
%%%%%%%%%%%%%%%%%%%%%%%%%%%%%%%%%%%%%%%%%%%%%%%%%%%%%%%%%%%%%
\chapter{関連研究}

\section{グループ1}

\section{グループ2}




%%%%%%%%%%%%%%%%%%%%%%%%%%%%%%%%%%%%%%%%%%%%%%%%%%%%%%%%%%%%%
%%%%%%%%%%%%%%%%%%%%%%%%%%%%%%%%%%%%%%%%%%%%%%%%%%%%%%%%%%%%%
\chapter{提案手法}

\section{設計指針}

\section{システム構成}

\section{ユーザインタフェース}

\section{アルゴリズム}




%%%%%%%%%%%%%%%%%%%%%%%%%%%%%%%%%%%%%%%%%%%%%%%%%%%%%%%%%%%%%
%%%%%%%%%%%%%%%%%%%%%%%%%%%%%%%%%%%%%%%%%%%%%%%%%%%%%%%%%%%%%
\chapter{評価実験}


\section{仮説}
評価実験を設計するにあたり以下3件の仮説を立てる.
\begin{itemize}
  \item 仮説1) .
  \item 仮説2) 
  \item 仮説3) 
\end{itemize}
この3件の仮説は,それぞれ以下の考察に基づき設定されている.
仮説1は...


\section{実験手順}




%%%%%%%%%%%%%%%%%%%%%%%%%%%%%%%%%%%%%%%%%%%%%%%%%%%%%%%%%%%%%
%%%%%%%%%%%%%%%%%%%%%%%%%%%%%%%%%%%%%%%%%%%%%%%%%%%%%%%%%%%%%
\chapter{結果と考察}


\section{実験1について}


\section{実験2について}





%%%%%%%%%%%%%%%%%%%%%%%%%%%%%%%%%%%%%%%%%%%%%%%%%%%%%%%%%%%%%
%%%%%%%%%%%%%%%%%%%%%%%%%%%%%%%%%%%%%%%%%%%%%%%%%%%%%%%%%%%%%
\chapter{まとめと展望}





\chapter*{謝辞}
\addcontentsline{toc}{chapter}{謝辞}

\bibliographystyle{junsrt}
\bibliography{ref.bib}


\end{document}
