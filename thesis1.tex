\documentclass[12pt,a4j]{jreport}
\setcounter{secnumdepth}{5}
\usepackage[dvipdfmx]{graphicx}
\usepackage{amsmath,amssymb}
%\usepackage{comment}
\usepackage{graphicx}
%\usepackage{here}
\usepackage{bm}
\usepackage{url}

\renewcommand{\baselinestretch}{1.5}

\renewcommand{\bibname}{参考文献}





\begin{document}


%%%%%%%%%%%%%%%%%%%%%%%%%%%%%%%%%%%%%
% 表紙
%%%%%%%%%%%%%%%%%%%%%%%%%%%%%%%%%%%%%
\begin{titlepage}

\begin{center}

    \vspace*{2cm}
    \Large 2025 年度 芝浦工業大学 工学部 情報工学科\\

    \vspace*{1.0cm}
    \Huge 卒 \qquad 業 \qquad 論 \qquad 文\\
    \vspace*{2.5cm}

    %TODO 編集 : 題目
    \Large 好みを反映した視認性の高いテキストエディタの配色を\\推薦する手法の提案および実装の\\引継ぎ研究
    
    \vspace{4cm}
    \begin{tabular}{ll}
        %TODO 編集 : 題目
        \vspace*{2mm}
        学籍番号 & \qquad $\mathbf{AL22020}$ \\
        \vspace*{2mm}
        氏\phantom{  }名 & \qquad 伊藤 \quad 柊生   \\
        \vspace*{2mm}
        指導教員           & \qquad 篠埜 \quad 功     \\
        \vspace*{2mm}
        提出日             & \qquad 2026年2月...日
    \end{tabular}
\end{center}
\end{titlepage}



\begin{abstract}
概要
\end{abstract}


{\makeatletter
\let\ps@jpl@in\ps@empty
\makeatother
\pagestyle{empty}
\tableofcontents
\clearpage}

\setcounter{page}{1} 
\pagestyle{plain}

%%%%%%%%%%%%%%%%%%%%%%%%%%%%%%%%%%%%%%%%%%%%%%%%%%%%%%%%%%%%%
% 序論 
%%%%%%%%%%%%%%%%%%%%%%%%%%%%%%%%%%%%%%%%%%%%%%%%%%%%%%%%%%%%%
\chapter{はじめに}

%\section{背景}
%背景で去年の卒論を引用して触れる。
Visual Studio Code や Vim など様々な項目を設定可能な高機能テキストエディタがいたるところで多く使われている. 
Stack overflow annual developer survey\cite{rateenv}の開発環境調査によると, 2024年5月の調査では, 質問に回答した開発者54175人のうち73.6\%がVisual Studio Code を使用している. 
これらのテキストエディタの多くは, プログラムの視認性を高めるために, 字句の種類ごとに色分けを行う機能を備えており, ユーザーはこれらの色設定を自由にカスタマイズすることが可能である。
テキストエディタにおける色設定の手段は大きく二つに分類できる。
第一の手段は、既に作成された配色データをダウンロードして利用する方法である。
Visual Studio Codeにおいては、テーマを拡張機能としてインストールすることで導入が可能であり、Visual Studio Marketplaceには、約14,000件に及ぶテーマ拡張機能が登録されている。
各拡張機能は一つ以上のテーマを提供しており、ユーザーはその中から好みの配色を選択する。
しかしながら、テーマの数が膨大であるため、希望に合致する配色を見つけ出す作業は容易ではない。
第二の手段は、ユーザー自身が個別に色設定を行う方法である。Visual Studio Codeにおいては、ワークベンチ関連の色設定項目が約780項目、さらにTextMateスコープに基づく色設定項目が100項目以上存在する。
なお、ワークベンチの色設定においては、図\ref{figure:vscode_workbench_color}に示すように、項目間に多くの依存関係が存在し、他の設定項目の色をもとに自動計算される項目も多数存在する。
そのため、実際にユーザーが手動で設定する必要がある項目は、ダークテーマにおいて約300項目、ライトテーマにおいて約303項目に留まる。

このように、全ての設定項目を網羅的に記述することにより、自分好みの配色テーマを作成することは可能であるが、設定すべき項目数が多く、加えてそれぞれに対して適切な色を選定する必要があるため、作業には非常に多大な労力を要する。



テキストエディタの配色の生成について、使えない配色が生成される場合があることと、
CUIのみを用いたツールのため、インターフェースの改良を行うことが課題である\cite{sotu24}

%\section{目的}


%\section{内容}
%卒論で何をするかも書く。

%\section{本論文の構成}



%%%%%%%%%%%%%%%%%%%%%%%%%%%%%%%%%%%%%%%%%%%%%%%%%%%%%%%%%%%%%
%%%%%%%%%%%%%%%%%%%%%%%%%%%%%%%%%%%%%%%%%%%%%%%%%%%%%%%%%%%%%
\chapter{関連研究}




%%%%%%%%%%%%%%%%%%%%%%%%%%%%%%%%%%%%%%%%%%%%%%%%%%%%%%%%%%%%%
%%%%%%%%%%%%%%%%%%%%%%%%%%%%%%%%%%%%%%%%%%%%%%%%%%%%%%%%%%%%%
\chapter{提案手法}

%%%%%%%%%%%%%%%%%%%%%%%%%%%%%%%%%%%%%%%%%%%%%%%%%%%%%%%%%%%%%
%%%%%%%%%%%%%%%%%%%%%%%%%%%%%%%%%%%%%%%%%%%%%%%%%%%%%%%%%%%%%
\chapter{実験}

\section{実験手順}

%%%%%%%%%%%%%%%%%%%%%%%%%%%%%%%%%%%%%%%%%%%%%%%%%%%%%%%%%%%%%
%%%%%%%%%%%%%%%%%%%%%%%%%%%%%%%%%%%%%%%%%%%%%%%%%%%%%%%%%%%%%
\chapter{実験結果および考察}


%%%%%%%%%%%%%%%%%%%%%%%%%%%%%%%%%%%%%%%%%%%%%%%%%%%%%%%%%%%%%
%%%%%%%%%%%%%%%%%%%%%%%%%%%%%%%%%%%%%%%%%%%%%%%%%%%%%%%%%%%%%
\chapter{まとめと今後の課題}

\chapter*{謝辞}
\addcontentsline{toc}{chapter}{謝辞}

\bibliographystyle{junsrt}
\bibliography{ref.bib}


\end{document}
